\documentclass[conference]{IEEEtran}

\usepackage[pdftex]{graphicx}
\graphicspath{ {./images/} }

\usepackage{amsmath}
\usepackage{amsfonts}
\usepackage{amssymb}
\interdisplaylinepenalty=2500

\usepackage{array}

\begin{document}
\title{Proposed Modem Design}
\author{\IEEEauthorblockN{Daryl Leon Wasden, Ph.D.\\
Signal Laboratories, Inc.}
\IEEEauthorblockA{Salt Lake City, Utah\\
Email: daryl@siglabs.com}}

\maketitle

\begin{abstract}
This document summarizes the design decisions/trade offs of the zeroth
iteration of the multicarrier modem used for distributed beamforming by
Signal Laboratories, Inc. All material should be considered confidential
and should only be used by authorized persons employed by Signal
Laboratories, Inc.
\end{abstract}

\section{Introduction}
\label{sec:intro}

In the context here, \emph{distributed beamforming} is defined as a
protocol for several wireless modems coordinating their transmissions
to boost the signal-to-noise ratio (SNR) at a receiving wireless modem
by pre equalizing their transmission sequences such that all (or most)
of the signals coherently combine at the intended receiving wireless modem.

This document is meant to act as a reference for the implementation
of the modem, and it will discuss design decisions, trade offs, and
provide some simulation and theoretical results to help guide the
system implementation. It is primarily concerned with the PHY layer
(although it may touch on MAC issues).

We now describe the structure of this document. Section~\ref{sec:hardware}
introduces the relevant hardware parameters that relate to our modem
design. The modem parameters are selected to complement the hardware
platform. Section~\ref{sec:FMTintro} presents the multicarrier modulation
scheme that will be employed in this version of the modem. This scheme was
chosen due to its similarity with the narrowband modem that has already been
developed, and a different choice (notably OFDM) may be employed
later if it is deemed advantageous. The modem parameters such as subcarrier
spacing, occupied bandwidth, etc. are also developed in this section.
Section~\ref{sec:chan} introduces the RF channel impairments under
consideration in the current iteration (i.e. multipath, doppler, etc).
Symbol recovery in the presence of RF impairments will be addressed in
the description of the demodulation procedures for the receiver in
Section~\ref{sec:rx}.

\section{Hardware}
\label{sec:hardware}

The hardware used to implement this modem design is a Circuit Card
Assembly (CCA) consisting of two printed circuit boards (PCBs), namely
a \emph{Graviton} PCB (Revision 2 or 3) with a \emph{Copper Suicide} PCB
(Revision 2) and a PC for providing software-related tasks.

\subsection*{Graviton}

The \emph{Graviton} PCB performs all tasks related to the RF analog
processing chain (amplification, attenuation, filtering, mixing, etc.)
as well as housing the analog-to-digital converter (ADC) and the
digital-to-analog converter (DAC).

The DAC on \emph{Graviton} is a Texas Instruments part DAC3484. It has
four channels (only 2 are available for use). It supports clock rates
up to 1.25 GSPS, but is currently clocked at 250 MSPS. It supports
digital upconversion and mixing internally, but we only make use of a
simple halfband interpolation filter and an inverse sinc functionality.
Other required interpolation and mixing is handled in the FPGA fabric.

The RF front end mixes the DAC signal output up from an IF of 35 MHz
to a carrier frequency of 915 MHz using the lower vestigial sideband.
This is noted here, because the DAC must provide an output centered
at the IF frequency with a frequency-reversed spectrum to match the
physical frequencies seen over-the-air. The RF front end should transmit
at a constant power level of approximately 1 W when the full dynamic
range of the DAC is used. While there are two RF signaling chains
available on \emph{Graviton}, we restrict our current design to use
only a single transmit/receive chain.

The RF front end contains a variable attenuator before it mixes the
signal at 915 MHz down to the 35 MHz IF. There is then a variable
gain and attenuation stage following the mixer before the signal
gets digitized by the ADC. The ADC on \emph{Graviton} is a Texas
Instruments part ADS42LB69. It has 16-bit inputs operating at
250 MSPS. One input

\begin{table}
\begin{tabular}{cc}
DAC Sampling Rate & 250 MSPS \\
ADC Sampling Rate & 250 MSPS \\
IF Frequency & 35 MHz \\
Sideband Used & Lower
\end{tabular}
\centering
\end{table}


\subsection*{Copper Suicide}



\subsection*{Personal Computer}



\section{Filtered Multitone Modulation}
\label{sec:FMTintro}

FMT is a modulation scheme that is essentially the same thing as a
traditional FDM system, except that it employs a polyphase filter
bank for efficient modulation and demodulation of the multicarrier
streams.

\subsection*{Transmit Mode}

\subsection*{Receive Mode}

\section{RF Channel Impairments}
\label{sec:chan}


\section{Transmitter Structure}
\label{sec:tx}

\section{Receiver Structure}
\label{sec:rx}

\section{Conclusion and Future Work}
\label{sec:conc}

\end{document}
