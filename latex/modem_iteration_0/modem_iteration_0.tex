\documentclass[conference]{IEEEtran}

\usepackage[pdftex]{graphicx}
\graphicspath{ {./images/} }

\usepackage{amsmath}
\usepackage{amsfonts}
\usepackage{amssymb}
\interdisplaylinepenalty=2500

\usepackage{array}

\begin{document}
\title{Proposed Modem Design}
\author{\IEEEauthorblockN{Daryl Leon Wasden, Ph.D.\\
Signal Laboratories, Inc.}
\IEEEauthorblockA{Salt Lake City, Utah\\
Email: daryl@siglabs.com}}

\maketitle

\begin{abstract}
This document summarizes the design decisions/trade offs of the zeroth
iteration of the multicarrier modem used for distributed beamforming by
Signal Laboratories, Inc. All material should be considered confidential
and should only be used by authorized persons employed by Signal
Laboratories, Inc.
\end{abstract}

\section{Introduction}
\label{sec:intro}

In the context here, \emph{distributed beamforming} is defined as a
protocol for several wireless modems coordinating their transmissions
to boost the signal-to-noise ratio (SNR) at a receiving wireless modem
by pre equalizing their transmission sequences such that all (or most)
of the signals coherently combine at the intended receiving wireless modem.

This document is meant to act as a reference for the implementation
of the modem, and it will discuss design decisions, trade offs, and
provide some simulation and theoretical results to help guide the
system implementation. It is primarily concerned with the PHY layer
(although it may touch on MAC issues).

We now describe the structure of this document. Section~\ref{sec:hardware}
introduces the relevant hardware parameters that relate to our modem
design. The modem parameters are selected to complement the hardware
platform. Section~\ref{sec:FMTintro} presents the multicarrier modulation
scheme that will be employed in this version of the modem. This scheme was
chosen due to its similarity with the narrowband modem that has already been
developed (at Signal Laboratories), and a different choice (notably OFDM)
may be employed later if it is deemed advantageous. The modem parameters
such as subcarrier spacing, occupied bandwidth, etc. are also developed
in this section. Section~\ref{sec:chan} introduces the RF channel
impairments under consideration in the current iteration (i.e. multipath,
doppler, etc). Symbol recovery in the presence of RF impairments will be
addressed in the description of the demodulation procedures for the
receiver in Section~\ref{sec:rx}.

\section{Hardware}
\label{sec:hardware}

The hardware used to implement this modem design is a Circuit Card
Assembly (CCA) consisting of two printed circuit boards (PCBs), namely
a \emph{Graviton} PCB (Revision 2 or 3) with a \emph{Copper Suicide} PCB
(Revision 2) and a PC for providing software-related tasks.

\subsection*{Graviton}

The \emph{Graviton} PCB performs all tasks related to the RF analog
processing chain (amplification, attenuation, filtering, mixing, etc.)
as well as housing the analog-to-digital converter (ADC) and the
digital-to-analog converter (DAC).

The DAC on \emph{Graviton} is a Texas Instruments part DAC3484. It has
four channels (only $2$ are available for use). It supports clock rates
up to $1.25$ GSPS, but is currently clocked at $250$ MSPS. It supports
digital upconversion and mixing internally, but we only make use of a
simple halfband interpolation filter and an inverse sinc functionality.
Other required interpolation and mixing is handled in the FPGA fabric.

The RF front end mixes the DAC signal output up from an IF of $35$ MHz
to a carrier frequency of $915$ MHz using the lower vestigial sideband.
This is noted here, because the DAC must provide an output centered
at the IF frequency with a frequency-reversed spectrum to match the
physical frequencies seen over-the-air. The RF front end should transmit
at a constant power level of approximately $1$ W when the full dynamic
range of the DAC is used. While there are two RF signaling chains
available on \emph{Graviton}, we restrict our current design to use
only a single transmit/receive chain.

The RF front end contains a variable attenuator before it mixes the
signal at 915 MHz down to the $35$ MHz IF. There is then a variable
gain and attenuation stage following the mixer before the signal
gets digitized by the ADC. The ADC on \emph{Graviton} is a Texas
Instruments part ADS42LB69. It has $16$-bit inputs operating at
$250$ MSPS. One input

\begin{table}
\begin{tabular}{|r|l|}
\hline
DAC Sampling Rate & $250$ MSPS \\
\hline
ADC Sampling Rate & $250$ MSPS \\
\hline
IF Frequency & $35$ MHz \\
\hline
Sideband Used & Lower \\
\hline
\end{tabular}
\centering
\vspace{0.125in}
\caption[Hardware Parameters]{Hardware Design Parameters}
\end{table}

\subsection*{Copper Suicide}

The \emph{Copper Suicide} board is a low cost ASIC prototyping platform that contains
$8$ ECP5 FPGAs from Lattice Semiconductor. Each FPGA has a $1$ GB of SDRAM that can be
used for large data storage. The FPGAs are all connected via a high speed bus interface
(the MIB Bus). In addition adjacent FPGAs have more than $34$ pins (exact amount?) for
high speed data transfers.

This layout means that the design needs to be properly divided up among the FPGAs, and
it is also preferably to use a stream processing with limited or no feedback in the design.
We are operating most of the DSP operations in a $125$ MHz clock domain with faster
clocks used when required.

\subsection*{Personal Computer}

The personal computer (PC) is connected to \emph{Copper Suicide} or \emph{Graviton}
Revision 2 through a high speed gigabit ethernet connection. The ethernet connection
is used for general purpose data transfers, debugging data transfers, and
command-and-control data transfer (i.e., reading and writing registers on the FPGAs,
controlling peripherals, etc.).
The PC is also used for some final signal processing related tasks such as encoding
and decoding of the data message, controlling the beamforming phase adjustment for
each symbol period, etc. Additionally, the PC controls reads/writes registers in
the FPGA during setup to put the modem into its initial state and monitors flag
registers to detect errors during normal operation.

\section{Filtered Multitone Modulation}
\label{sec:FMTintro}

Filtered Multitone (FMT) is a modulation scheme that is essentially the same thing
as a traditional multi-user frequency division multiplexing (FDM) systems. In
traditional FDM, multiple distinct users encode their data on non-overlapping
frequency bands (often called channels or carriers). In FMT, each frequency band
is used by the same user to increase the aggregate transmission rate of that
user. In FMT, each non-overlapping band is called a \emph{subchannel} or
\emph{subcarrier} to emphasize that it is part of a larger whole.

For efficient modulation and demodulation, a polyphase filter bank is used with
a prototype filter that is a square root Nyquist filter at the transmitter and
receiver. This filter bank takes the place of traditional pulse shaping and
matched filtering in single carrier narrowband modems.

The majority of the design operates in a $125$ MHz clock domain, but in order
to have multiple clock cycles per sample period (for resource sharing in the FPGA),
we operate on a signal that is decimated by $8$. Between the decimation filters
and the filter bank, our sampling rate is therefore $250 \mbox{MHz} / 8 = 31.25$ MHz.
We use a filter bank with an FFT size of $1024$ on both the transmit and receive.
The baud rate for each subcarrier signal is $31.25 \mbox{MHz} / 2048 = 15.287890625$ kHz.
The transmitter filter bank (TXFB) is designed such that the input rate should come in
at $1024 \mbox{channels} \times 15.287890625 \mbox{kHz per channel}$ whereas the output
of the receiver filter bank {RXFB} is designed to be oversampled by a factor of $2$. This
is designed this way so that software can perform a fractionally-spaced equalizer and to
ease resampling for timing adjustment if necessary.

\begin{table}
\begin{tabular}{|r|l|}
\hline
TXFB Input Sampling Rate per Channel & $15.287890625$ kHz \\
\hline
TXFB Output Sampling Rate & $31.25$ MHz \\
\hline
RXFB Input Sampling Rate & $31.25$ MHz \\
\hline
RXFB Output Sampling Rate per Channel & $30.517578125$ kHz \\
\hline
Subcarrier Spacing & $30.517578125$ kHz \\
\hline
\end{tabular}
\centering
\vspace{0.125in}
\caption[FMT Parameters]{FMT design parameters}
\end{table}

\subsection*{Transmit Mode}

The TXFB is efficiently implemented using a synthesis polyphase filter bank.
% Cite vaidyanathan, harris, farhang
The prototype filter used for pulse shaping at the transmitter is a square root
Nyquist pulse shape designed with a baud interval of $2048$ samples and a
roll-off of $0.5$. The purpose of the filter bank is to generate a sampled version of
\begin{equation}
x(t) = \sum_{k=0}^{N-1} \sum_{m=-\infty}^{+\infty} s[m, k] p(t - m T_b) e^{j \frac{2 \pi t k f_a}{T_b}}
\end{equation}
where the sampled version replaces $t$ with $n T_s$ and we choose $\frac{T_s f_a}{T_b} = \frac{1}{N}$
and $T_b = 2 N T_s$ so that the expression simplifies to
\begin{equation}
x(n T_s) = \sum_{k=0}^{N-1} \sum_{m=-\infty}^{+\infty} s[m, k] p(n T_s - 2 m N T_s) e^{j \frac{2 \pi n k}{N}}.
\end{equation}
Swapping the order of summations and pulling the pulse shaping filter out of the inner summation
brings us to the form
\begin{equation}
x(n T_s) = \sum_{m=-\infty}^{+\infty} p(n T_s - 2 m N T_s) \sum_{k=0}^{N-1} s[m, k] e^{j \frac{2 \pi n k}{N}}.
\end{equation}
This form suggests the efficient implementation that we will use. For each fixed symbol sequence $m$,
we compute the IFFT of the $N$ subcarrier symbols and pass that sequence through a polyphase decomposition
of our sampled pulse shaping filter $p(n T_s)$ and pass them through a network of polyphase filters. For
our system, we are using $N=1024$.
% TODO: Put a block diagram here

\subsection*{Receive Mode}

The RXFB is efficiently implemented using an analysis polyphase filter bank.
The prototype filter is matched to the TXFB prototype filter (by conjugation and
time-reversal). In the case of a real symmetric prototype filter, the matched
version is identical to the original pulse shape.

\section{RF Channel Impairments}
\label{sec:chan}

\subsection*{Thermal Noise}
We consider a thermal noise model that is additive white gaussian noise (AWGN).
That is, the receive signal is perturbed by noise that is constant across the
operating band (white) and follows a gaussian distribution. This can be motivated
from the statistical mechanics of charged particles bashing into each other on an
atomic scale. The thermal noise density is given as $N_0 = k T$ where $k$ is
Boltzmann's constant $k = 1.38 \times 10^-23\:\mbox{J/K}$ and $T$ is the
temperature in Kelvin. At room temperature (i.e., $T = 290\:\mbox{K}$),
$N_0 \approx -174\:\mbox{dBm/Hz}$.

If the signal $y(t)$ represents the received signal and $x(t)$ is the transmitted
signal, then we can represent an AWGN $v(t)$ through the equation
\begin{equation}
y(t) = x(t) + v(t).
\end{equation}

\subsection*{Local Oscillator Phase Noise}
No oscillator is perfect. Even very good oscillators exhibit some phase noise.
Phase noise manifests itself as random fluctuations about the desired phase. For
a local oscillator (LO) designed to operate at $f_c\:\mbox{Hz}$, this can be
expressed as
\begin{equation}
x(t) = x_{BB}(t) \exp\left(j 2 \pi \left( f_c t + v_c(t) \right)\right)
\end{equation}
where $x(t)$ is the signal to be transmitted at the antenna, $x_{BB}(t)$
represents the baseband signal (I and Q), and finally
$\exp\left(j 2 \pi f_c t + v_c(t)\right)$ modulates the baseband signal
to the carrier or IF frequency $f_c$.

These random phase fluctuations typically manifest themselves as phase
rotations in the constellation diagram. This results in the points
being smeared in a circular direction. Points with a higher amplitude
are smeared more than points with a smaller amplitude (since the arc
length of the angle is larger).

In our system, we make the assumption that our clocks are insanely
good. Under this assumption, we ignore the effects of LO phase noise.

\subsection*{Sampling Clock Jitter}
In addition to phase noise on the local oscillator used for RF up/down
conversion, there is also potential for jitter in the phase of the
sampling clock. In our system, the sampling clock and the mixer LO
are derived from the same crystal. For that reason, we expect the
jitter on these two to be approximately equal (and highly correlated
if the processing delays are taken into account). In traditional
model designs, the phase noise from the LO and the phase noise
introduced from the sampling clock jitter are considered separately.

As before, our system has insanely good clocks so we assume the
sampling clock jitter is near zero and ignore the effects of jitter-induced
phase noise.

\subsection*{Carrier Frequency Offset}
In addition to random noise, it is often the case that the LO at the
transmitter and the LO at the receiver contain some fixed offset from
each other. This shows up in the receiver baseband signal as a slightly
modulated version of the desired signal,
\begin{equation}
\hat{x}_{BB}(t) = x_{BB}(t) \exp\left(-j 2 \pi \Delta_f t\right),
\end{equation}
where $\Delta_f = f'_c - f_c$, $f'_c$ is the carrier frequency
at the receiver, and $f_c$ is the carrier frequency at the transmitter.
We have neglected any additive noise to improve the clarity of the
equation.

So far, we have observed offsets from $-100$ to $100$ Hz. We assume
that we will not have to deal with offsets greater than this.

\subsection*{Sampling Frequency Offset}
A sampling clock offset between the transmitter and receiver has a
similar, but slightly more complicated effect on the observed signal.
It causes the receiver to observe a time-scaled version of the signal,
\begin{equation}
\hat{x}_{BB}(t) = x_{BB}(t') \exp\left(-j 2 \pi f_c \left(t - t'\right)\right),
\end{equation}
where $t' = (1 + \gamma) t$ and $\gamma$ is a small parameter in the
range from $-\Delta_t$ to $\Delta_t$ and $\Delta_t$ is typically much
smaller than $1$. So, we can write this expression in terms of $\gamma$
as
\begin{equation}
\hat{x}_{BB}(t) = x_{BB}(t') \exp\left(-j 2 \pi f_c \gamma t\right).
\end{equation}
This shows that the new signal is a time-scaled and slightly modulated
version of the desired signal. The modulation is at an offset of $-f_c \gamma$.

Notice that things are even more complicated when considering Carrier
Frequency Offset (CFO) and Sampling Frequency Offset (SFO) together, since
we obtain
\begin{equation}
\hat{x}_{BB}(t) = x_{BB}(t') \exp\left(-j 2 \pi \left(f_c t - f'_c t'\right)\right).
\end{equation}
Now, we can apply the fact that $f'_c = f_c + \Delta_f$ and $t' = (1 + \gamma) t$
to obtain
\begin{equation}
\hat{x}_{BB}(t) = x_{BB}(t') \exp\left(-j 2 \pi \left(f_c t - \left(f_c + \Delta_f\right) \left(1 + \gamma\right) t \right)\right)
\end{equation}
or if we simplify,
\begin{equation}
\hat{x}_{BB}(t) = x_{BB}(t') \exp\left(-j 2 \pi f_c \gamma t\right) \exp\left(-j 2 \pi \Delta_f \left(1 + \gamma\right) t\right).
\end{equation}
Notice that in this case, the CFO and SFO both contribute to a CFO-like phase
error, and the time-scaling associated with the received signal is still present.
We again assume that our oscillators are near enough so that a CFO estimate is
necessary, but an SFO estimate will have a negligible effect over the duration
of the transmission before another estimation event occurs.


\subsection*{Multipath Channel}

\subsection*{Slow and Fast Fading}

\subsection*{Doppler Effects}

\subsection*{Shadowing (Large-scale Fading)}

\section{Transmitter Structure}
\label{sec:tx}

\section{Receiver Structure}
\label{sec:rx}

\section{Conclusion and Future Work}
\label{sec:conc}

\end{document}
